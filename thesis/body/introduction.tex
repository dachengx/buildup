\section{引言}

$\gamma$ 射线和 X 射线的衰减常数和与物质的作用截面的研究被很多基础科学与应用科学涉及,在一些领域,如核医学诊疗与辐射防护更具有极其重要的意义\cite{kerur_mass_2009}。在核医学诊疗中,一般利用辐射如 $\gamma$ 射线对肿瘤组织等治疗位点进行辐照以达到杀死有害组织的目的。因为人体组织的不均匀性和人体形状的不均匀性,若不进行合理的屏蔽,则射线对正常组织的伤害不可忽略。保护正常组织以及精准破坏有害组织是医学物理的主要课题。特定辐射和屏蔽体对剂量的影响的研究对这些目标的实现非常重要\cite{ding_need_2007}。安全且有效的只有在正常组织接受的剂量处于安全范围内时,新的辐射治疗方法才能实际应用。$\pm5\%$ 的剂量差异就会导致肿瘤治疗中的正常组织损伤风险\cite{dische_precision_1993}。

组织接受的辐射不均匀的一个原因是人体形状的不均匀性。为了解决这个问题,一般将合适的材料制作成补偿器,作为特定组织的等效物,来对厚度不足的组织进行补偿\cite{foster_influence_2009}。同时因为需要进行治疗的组织只在人体中占有比较小的体积,所以经常使用准直器来对射线进行约束以达到精准治疗的目的。这这些使用场景中,都需要合适的辐射屏蔽材料\cite{zhu_intensity-modulated_2005}。

为了达到这些目的,需要对屏蔽物质的特性和其与辐射作用的过程进行研究。对于窄束 $\gamma$ 射线,使用衰减公式\eqref{eq:attenuation}对物质的屏蔽作用进行描述\cite{foster_influence_2009}:
\begin{align}
    N &= N_0 e^{-\mu d} \label{eq:attenuation} \\
    N &= N_0 B e^{-\mu d} \label{eq:buildup}
\end{align}
其中 $N_0$ 是定方向窄束中的发射粒子个数,$d$ 为屏蔽体的厚度,$N$ 为经过 $d$ 屏蔽后剩余的粒子个数,$\mu$ 为物质和辐射作用的线衰减常数,单位一般为 $\si{cm^{-1}}$。$\mu$ 与射线种类、能量、屏蔽物质的密度与成分均有关。

在宽束情况下,公式\eqref{eq:attenuation}需要被进行修正,修正结果为\eqref{eq:buildup}。其中 $B$ 为累计因子。累计因子在射线准直、组织补偿,以及辐射屏蔽中均有重要意义。累计因子的大小对于屏蔽后辐射的品质很重要,对辐射治疗中的组织剂量计算非常重要。光子与物质的相互作用一般被认为是,在每一次作用中光子都会被直接消耗,但实际上,在康普顿散射(Compton scattering)中,光子在与电子相互作用后并不会被直接吸收,而是会继续存在并可能再次发生相互作用。辐射防护中大多遇到的宽束问题,这时射线束较宽,准直性较差,射线穿过的物质层也相当厚。在这种情况下,收到散射的光子经过二次甚至多次散射厚仍有可能穿过物质,达到所考察的空间位置上。在所考察点的位置上观察到的不仅包括那些未经相互作用的入射光子,而且还有经多次散射后的散射光子。累计因子带来的效应有时很显著,所以对其的研究很有必要。

MCP-96 是一种低熔点合金,其成分为:$52.5\%$ 铋(Bi)、$32\%$ 铅(Pb)、$15.5\%$ 锡(Sn),密度为 $9.72\si{g/cm^3}$。MCP-96 合金因为其低熔点、屏蔽作用显著的特点,经常被应用于辐射防护。值得注意的是,MCP-96 不含有低熔点合金常见的具有高毒性的镉(Cd)元素。表\ref{tab:mcp96}列举了 MCP-96 合金的主要物理特性\cite{hopkins_linear_2012}。

\begin{table}[H]
    \centering
    \caption{MCP-96 合金的物理特性}
    \label{tab:mcp96}
    \begin{tabular}{l|r}
    \toprule
    物理量 & 物理量的数值 \\ 
    \hline
    熔点/$\si{K}$ & $369$ \\
    \hline
    密度/$\si{g\cdot cm^{-3}}$ & $9.72$ \\
    \hline
    比热/$\si{J\cdot kg^{-1}\cdot K^{-1}}$ & $151$ \\
    \hline
    导热率/$\si{W\cdot m^{-1}\cdot K^{-1}}$ & $12.5$ \\
    \hline
    电阻率/$\si{\mu\ohm\cdot cm}$ & $71.4$ \\
    \bottomrule
    \end{tabular}
\end{table}

MCP-96 在辐射防护中的应用,对其线衰减常数和累计因子的研究提出了要求。遗忘的研究主要集中在 MCP-96 与辐射相互作用的实验研究上\cite{hopkins_linear_2012}。

蒙特卡洛方法(Monte Carlo method)是研究辐射与物质的相互作用的主要常见方法之一。蒙特卡罗方法使用已知的辐射粒子与物质的相互作用过程为输入量,在模拟中生成符合相互作用的随机过程,通过增大模拟的事件数量,最终得到大量粒子与物质相互作用的集体规律,在金融工程、计量经济、生物医学、计算物理等领域应用广泛。在辐射防护领域,是基于实验的辐射与物质相互作用研究的有效补充。通过蒙特卡罗方法,可以检验辐射与物质相互作用实验,如测量合金的线衰减常数实验的正确性。

本篇论文中,使用欧洲核子研究组织(CERN)主导开发的粒子物理模拟软件包 \href{https://geant4.web.cern.ch/}{Geant4}\cite{agostinelli_geant4simulation_2003,allison_geant4_2006},进行对 MCP-96 合金的线衰减常数和累计因子的模拟研究。首先验证以论文\cite{hopkins_linear_2012}为代表的 MCP-96 对 $\gamma$ 射线的屏蔽作用的实验研究结果,之后引入更多更细致的实验参数设定,开展更加精细化的研究。