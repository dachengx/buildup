\section{讨论}

从数据比较上看,线衰减常数 $\mu$ 的模拟结果与实验结果的差异较小,而累计因子与 MCP-96 衰减器的厚度、$\gamma$ 射线能量、衰减器与探测器之间距离的关系的模拟结果与实验结果差异较大。

线衰减常数 $\mu$ 与 $\gamma$ 射线能量 $E$ 关系的是实验结果与模拟结果的差异应来自于模拟中的粒子模拟个数($10^5$)偏少,这是因为可以观察到 $mu$ 与 $E$ 在考察的主要能区($0\sim2\si{MeV}$),随着 $E$ 增大,模拟结果与实验结果均 $\mu$ 减小,这是由于当 $\gamma$ 射线能量增加时,光电效应的截面减小,而康普顿散射的截面增大但并不能抵消光电效应的减小程度,且实验结果与模拟结果的数量差异很小。

其中累计因子与 MCP-96 衰减器的厚度、$\gamma$ 射线能量的关系的差异可能主要实验中的本底控制不可能绝对完美导致的。当 $\gamma$ 射线信号数量一定时,本底越大,得到的累计因子越小。若不加入准直器时的信号计数为 $N_1$,加入准直器时的信号计数为 $N_2$,本底为 $b$,则 $N_2 > N_1$,且累计因子可以表示为:

\begin{align}
    B &= \frac{N_2 + b}{N_1 + b}
\end{align}

当 $b$ 从 $0$ 逐渐增大时,$B$ 逐渐减小以至于趋于 $1$。这在一定程度上也可以解释衰减器与探测器之间距离 $d'$ 与累计因子 $B$ 变化中模拟结果偏大,实验结果偏小的原因。